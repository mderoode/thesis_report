\section*{Case Studies}

This main subject of this thesis is inter-case resource patterns and consists of the following parts: role mining, resource bottleneck and work-allocation patterns analysis. Role Mining aims to identify the roles of resources based on their behaviour. Resource Bottleneck analysis looks at the availability and workload of resources and aims to find bottlenecks in a process by analysing resources and roles. Finally, work-allocation patterns aims to check if resources and roles follow certain patterns, such as for example: handling tasks in a First-in-First-out (FiFo) or Last-in-First-out (LiFo) manner. The analysis, furthermore, looks at why resources sometimes deviate from their common behaviour. The following sections describe how a case study can be approached for the subjects in this thesis. 

% There are two main reasons for conducting case studies, namely: to ensure that the analysis can be use in practise and to validate the results of the analysis techniques. 



\subsection*{Role Mining}
Role Mining only requires an event log where each event is related to a resource.  The case study starts with a demo of the proof-of-concept for the client including an interview to gather domain knowledge. Next, a role mining analysis is conducted by the student or by the customer and the results are shared with a domain expert. The domain expert can validate the findings of the role mining analysis by comparing the results with its organizational structure and domain knowledge. If the student is not authorized to use the client data, the role mining analysis framework is explained to the client and the client can perform the analysis theirself. 


% There are two possible methods to perform  role mining in a case study depending on whether the student and consultant are authorized to use the client data. In both cases, the case study starts with a demo of the proof-of-concept for the client including an interview to gather some domain knowledge about the process. 

% If the student/consultant is authorized to use the client data, a role mining analysis can be conducted by the student/consultant and the results are shared with a domain expert. The domain expert can validate the findings of the role mining analysis by comparing the results with its organizational structure and domain knowledge. This process might be iterative, in other words: the analysis could be adapted based on feedback of the domain expert to optimize the analysis. 

% If the student is not authorized to use the client data, the role mining analysis framework is explained to the client after the demo. The proof-of-concept is than deployed such that it is able to use the client's data. The client can then perform the role mining analysis themself including a validation of the results (e.g. comparing the results against their security policy). Afterwards, the client is interviewed again and the findings of the role mining are shared.

\subsection*{Resource Bottleneck Analysis}
% The resource bottleneck analysis case study is slightly different from the role mining case study because it consists of a set of dashboards instead of an algorithm. 

The bottleneck analysis requires an event log containing events which are related to resources and preferably event start and event end times. The case study starts with a demo of the proof-of-concept including an interview to gather domain knowledge. The client, furthermore, gets a training/workshop on how to use the dashboards and the proof-of-concept is deployed such that it is able to use the client's data. The client can then perform the resource bottleneck analysis on its own data and the findings are afterwards evaluated. 

% The findings include the usefulness of the analysis, the correctness of the analysis and the user experience. 

% Furthermore, it is optional that the resource roles are known and the resource availability is known. 



% If the student/consultant are authorized to use the client data, they can walkthrough the bottleneck analysis together with the client. 


\subsection*{Work-allocation patterns}
The work-allocation pattern case study is similar to the resource bottleneck analysis case study. However, this analysis can be tailored to the client's requirements by analysing patterns which are specific to the client. The standard work-allocation patterns include whether resources follow FiFo, LiFo, Shortest-Job-First, Closest-Deadline-First prioritization patterns. However, the client can define their own patterns which can be implemented specifically for the customer.  The data required for this case study depends on the client's work-allocation patterns. The standard patterns require an event log which contains a related resource per event including event start and event end times. 

\chapter*{Inter-case resource patterns}
This main subject of this thesis is inter-case resource patterns and consists of the following parts: role mining, resource bottleneck and work-allocation patterns analysis. Role Mining aims to identify the roles of resources based on their behaviour. Resource Bottleneck analysis looks at the availability and workload of resources and aims to find bottlenecks in a process by analysing resources and roles. Finally, work-allocation patterns analysis aims to check if resources and roles follow certain patterns, such as for example: handling tasks in a First-in-First-out (FiFo) or Last-in-First-out (LiFo) manner. The analysis, furthermore, looks at why resources sometimes deviate from their common behaviour. 

The case study starts with a demo of the proof-of-concept for the client including an interview to gather domain knowledge. Next, if the student is authorised to use the client data, an analysis using the proof-of-concept is conducted by the student together with the consultant/client. The results are shared with a domain expert. The domain expert can validate the findings of the analysis by comparing the results with its organizational information and domain knowledge. If the student is not authorized to use the client data, the role mining analysis framework is explained to the client and the client can perform the analysis theirself. The client can then validate whether the framework and its results together with the student.

For the case study, an event log including event start and event end times are required. Furthermore, each event should be related to a resource. Finally, there should be a domain expert available who can validate the findings. 


% The case study starts again with a demo of the proof-of-concept including an interview to gather some domain knowledge about the process. Furthermore, the client can  request custom work-allocation patterns. After the custom work-allocation patterns are implemented, the client gets a training/workshop about how to use the dashboards and the proof-of-concept is deployed such that it is able to use the client's data. The client can then perform the  analysis on its own data and the findings are afterwards evaluated. The findings include the usefulness of the analysis, the correctness of the analysis and the user experience. If the student/consultant are authorized to use the client data, they can walk-through the analysis together with the client. 
